\documentclass[12pt]{article}


\usepackage[utf8]{inputenc}
\usepackage[russian]{babel}
\usepackage{amsmath}
\usepackage{amssymb}
\usepackage{geometry}
\geometry{top=2cm} %поле сверху
\geometry{bottom=2cm} %поле снизу
\geometry{left=2cm} %поле справа
\geometry{right=2cm}
\usepackage{graphicx}


\renewcommand{\rmdefault}{fac}

\begin{document}
\title{SmallIDE}
\author{Кирилл Павленко, Антон Шумаков}
\date{\today}
\maketitle
\newpage

\section{Цели}
Целью является создание интегрированной среды разработки (IDE) для комфортного программирования на языке Java. Задача имеет прикладной характер.

\section{Общая идея}
Мы ставим следующую задачу: создать IDE с максимально гибким и легко изменяемым интерфейсом для программистов Java. Чем наша среда разработки отличается от любой другой? В чем ее принцпиальное отличие? Идея заключается в том, что каждый функциональный элемент самой среды является модулем. Любой сторонний программист сможет легко переписать, модифицировать каждый модуль или добавить свой собственный модуль по своим предпочтениям. Это касается не только графических элементов, но и любых других, по нашей задумке изменениям может подвергнуться любой функциональный элемент. Будет реализован простой способ добавления модулей; будет доступен специальный конфигурационный файл, в который можно прописать названия требуемых модулей. Наша схема позволит различным программистам обмениваться между собой модулями и сборками, что, в перспективе, поможет автономно существовать нашей среде разработки.

\section{Структура программы}

(Подробное описание основных элементов IDE см. в разделе $"$Приложение$"$ .)



Основой структурный элемент программы --- модуль. Модулем мы называем всякий класс, унаследованный от класса Module, выполнящий конкретную подзадачу нашей IDE.
 
Также мы реализуем три независимых, не являющихся модулями, компонента программы:
\begin{itemize}
\item Главное окно
\item Система обмена сообщениями
\item Система настройки
\end{itemize}
Эти три объекта являются единственными неизменяемыми элементами IDE. 

Главное окно представляет из себя поле для размещения графических элементов и выполняет функцию некоего $"$каркаса$"$ программы. 

Система сообщений необходима для взаимосвязи модулей между собой. 

Система настройки --- структура для сохранения параметров модулей.

Все остальные элементы программы --- отдельные модули. Некоторые из них --- графические --- взаимодействуют с пользователем через главное окно. Другие, не являющиеся графическими элементами, могут $"$общаться$"$ друг с другом посредством системы сообщений. Пользователь может иметь к ним доступ через систему настройки. 


Ниже представлена схема, демонстрирующая структуру нашей программы.


%\includegraphics[width=0.7\textwidth]{drawing.pdf}

\section{О реализации}

Проект будет полностью написан на языке Java, с использованием системы контроля версий --- git. Весь код будет доступен в свободном доступе на GitHub.com, там же будет отображаться прогресс нашей работы.

Ниже указаны приблизительные сроки работы.
\begin{itemize}
\item 5 марта. Начало работы.
\item 12 марта. Написание главного окна и его методов 
\item 22 марта. Реализация системы сообщений
\item 30 марта. Реализация методов системы настройки.
\item 10 апреля. Покрытие тестами существующей части проекта
\item 15 апреля. Первая сборка IDE, минимальный пакет.
\item 25 апреля. Реализация конструктора для добавления модулей.
\item В остальное время будет произведена работа над графикой, по возможности будет расширен стандартный пакет.
\end{itemize}

Задача разделяется на две части: написание ядра программы и написание графической части. Первым займется Кирилл Павленко, а вторым --- Антон Шумаков.

\section{Программное обеспечение}
Наша IDE кроссплатформенная, требует предустановленной Java. 

\section{Приложение}
Принципиальное описание основных элементов нашей программы.

\subsection{Модуль}
Название класса. Module

Описание. Абстрактный класс для описания модуля. Содержит несколько методов, осуществляющих взаимодействие модуля с ядром IDE.


\subsection{Система сообщений}

{\bf Название класса.} RussianPost

\textit{Описание.} Используется два типа ящиков: входящие и исходящие. Первые служат для передачи сообщений конкретным модулям, а вторые обеспечивают массовую рассылку всем подписавшимся на ящик.

\textbf{Основные методы.} 

\begin{enumerate}

\item void addIncPostBox (String name, MsgRcvr mr)

Добавление исходящего ящика. В параметры передается имя ящика и получатель типа MsgRcvr
\item OutBox addOutPostBox(String name)

Добавление входящего ящика. В параметрах --- имя ящика.
\item void subscribe(String name, MsgRcvr self)

Подписка на существующий ящик. В аргументы передается имя ящика, и объект-подписчик типа MsgRcvr 
\item void send(String name, Object msg)

Передача сообщения на исходящий ящик. В аргументы передается имя ящика и сообщение.
\end{enumerate}

\subsection{Система настройки модулей}


Название класса. Configuration

Описание. Позволяет модулям хранить необходимые им данные и получать их по необходимости.
\begin{enumerate}
\item void add(String self, String objname, Object forserialize)

Добавление объекта для храниения. В аргументах: имя владельца, имя объекта, объект

\item Object get(String name, String objname)

Получить объект. Аргументы: имя владельца, имя объекта.
\item boolean register(String self)

Зарегистрироваться в системе настроек. Аргумент: имя модуля.
\end{enumerate}

\subsection{Главное окно}

{\bf Название класса.} MainFrame

\textit{Описание.} Предоставляет всем желающим графическим модулям место в окне. Само окно условно разделено на четыре панели с вкладками (TOP, BOTTOM, LEFT, RIGHT). Запрашивая место в окне модуль получает место на одной из таких панелей в виде отдельной вкладки.

\textbf{Основные методы.} 

\begin{enumerate}

\item int addMenu(String path, Module m, KeyStroke binding) throws ConflictException

Добавление элемента меню с адресом path, принадлежащего модулю m, и запускаемого по клавиатурному сокращению binding.
\item int addButton(Icon icon, Module m)

Добавляет на панель инструментов новую кнопку с иконкой icon, принадлежащую модулю m.
\item JPanel addGraphicsModule(GraphicsModule gm, int where, String name)

Выделяет графическому модулю gm вкладку с названием name в одном из четырёх возможных мест (задаётся параметром where).
\end{enumerate}

\subsection{Редактор текста}

{\bf Название класса.} TextEditor

\textit{Описание.} Представляет текстовое поле. Является основным модулем. Не зависим от других модулей.

\textbf{Почтовые ящики.} 

\begin{enumerate}

\item название: SetKeys; тип: входящий; тип сообщения: KeyBindingsInitializer;

Установка собственных клавиатурных сокращений.
\item название: ShowPopupMenu; тип: входящий; тип сообщения: PopupMenu;

Выбрасывает всплывающее меню под текущим положением курсора.
\item название: SetDocument; тип: входящий; тип сообщения: TextFile;

Установка текущего, отображаемого редактором, текстового файла.
\item название: OpenFile; тип: входящий; тип сообщения: File;

Открыть файл.
\item название: ActiveDocument; тип: исходящий; тип сообщения: TextFile;

Рассылается TextFile, ассоциированный с текущей вкладкой. Рассылка происходит при смене вкладки.

\item название: OpenDocument; тип: исходящий; тип сообщения: TextFile;

Рассылается TextFile. Рассылка происходит при открытии нового файла.
\item название: CloseDocument; тип: исходящий; тип сообщения: TextFile;

Рассылается TextFile. Рассылка происходит при закрытии файла.

\end{enumerate}

\subsection{Менеджер проектов}

{\bf Название класса.} SingleFileProjectManager

\textit{Описание.} Предоставляет список открытых файлов. Зависим от текстового редактора и компилятора. Предоставляет кнопки создания нового файла,запуска и компиляции.

\textbf{Почтовые ящики.} 

Отсутствуют.

\subsection{Компилятор}

{\bf Название класса.} Compiler

\textit{Описание.} Предоставляет текстовой поле куда будут перенаправляться стандартные потоки ввода-вывода запущенного приложения и сторонней программы компиляции. Независимый модуль.

\textbf{Почтовые ящики.} 

\begin{enumerate}

\item название: Compile; тип: входящий; тип сообщения: File;

Запуск компиляции данного файла.

\item название: Run; тип: входящий; тип сообщения: File;

Запуск скомпилированного файла.

\end{enumerate}

\subsection{Подсветка синтаксиса}

{\bf Название класса.} Highlighter

\textit{Описание.} Загружает из конфигурационного файла colors описание подсветки для текстового редактора Vim в упрощённом варианте. Подсвечивает ключевые слова и комментарии. Зависит от текстового редактора.

\textbf{Почтовые ящики.} 

Отсутствуют.


\end{document}