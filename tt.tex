\documentclass[12pt]{article}


\usepackage[utf8]{inputenc}
\usepackage[russian]{babel}
\usepackage{amsmath}
\usepackage{amssymb}
\usepackage{geometry}
\geometry{top=2cm} %поле сверху
\geometry{bottom=2cm} %поле снизу
\geometry{left=2cm} %поле справа
\geometry{right=2cm}
\usepackage{graphicx}


\renewcommand{\rmdefault}{fac}

\begin{document}
\title{SmallIDE}
\author{Кирилл Павленко, Антон Шумаков}
\date{\today}
\maketitle
\newpage

\section{Цели}
Целью является создание интегрированной среды разработки (IDE) для комфортного программирования на языке Java. Задача имеет прикладной характер.

\section{Общая идея}
Мы ставим следующую задачу: создать IDE с максимально гибким и легко изменяемым интерфейсом для программистов Java. Чем наша среда разработки отличается от любой другой? В чем ее принцпиальное отличие? Идея заключается в том, что каждый функциональный элемент самой среды является модулем. Любой сторонний программист сможет легко переписать, модифицировать каждый модуль или добавить свой собственный модуль по своим предпочтениям. Это касается не только графических элементов, но и любых других, по нашей задумке изменениям может подвергнуться любой функциональный элемент. Будет реализован простой способ добавления модулей; будет доступен специальный конфигурационный файл, в который можно прописать названия требуемых модулей. Наша схема позволит различным программистам обмениваться между собой модулями и сборками, что, в перспективе, поможет автономно существовать нашей среде разработки.

\section{Структура программы}

(Подробное описание основных элементов IDE см. в разделе $"$Приложение$"$ .)



Основой структурный элемент программы --- модуль. Модулем мы называем всякий класс, унаследованный от класса Module, выполнящий конкретную подзадачу нашей IDE.
 
Также мы реализуем три независимых, не являющихся модулями, компонента программы:
\begin{itemize}
\item Главное окно
\item Система обмена сообщениями
\item Система настройки
\end{itemize}
Эти три объекта являются единственными неизменяемыми элементами IDE. 

Главное окно представляет из себя поле для размещения графических элементов и выполняет функцию некоего $"$каркаса$"$ программы. 

Система сообщений необходима для взаимосвязи модулей между собой. 

Система настройки --- структура для сохранения параметров модулей.

Все остальные элементы программы --- отдельные модули. Некоторые из них --- графические --- взаимодействуют с пользователем через главное окно. Другие, не являющиеся графическими элементами, могут $"$общаться$"$ друг с другом посредством системы сообщений. Пользователь может иметь к ним доступ через систему настройки. 


Ниже представлена схема, демонстрирующая структуру нашей программы.


%\includegraphics[width=0.7\textwidth]{drawing.pdf}

\section{О реализации}

Проект будет полностью написан на языке Java, с использованием системы контроля версий --- git. Весь код будет доступен в свободном доступе на GitHub.com, там же будет отображаться прогресс нашей работы.

Ниже указаны приблизительные сроки работы.
\begin{itemize}
\item 5 марта. Начало работы.
\item 12 марта. Написание главного окна и его методов 
\item 22 марта. Реализация системы сообщений
\item 30 марта. Реализация методов системы настройки.
\item 10 апреля. Покрытие тестами существующей части проекта
\item 15 апреля. Первая сборка IDE, минимальный пакет.
\item 25 апреля. Реализация конструктора для добавления модулей.
\item В остальное время будет произведена работа над графикой, по возможности будет расширен стандартный пакет.
\end{itemize}

Задача разделяется на две части: написание ядра программы и написание графической части. Первым займется Кирилл Павленко, а вторым --- Антон Шумаков.

\section{Программное обеспечение}
Наша IDE кроссплатформенная, требует предустановленной Java. 

\section{Приложение}
Принципиальное описание основных элементов нашей программы.

\subsection{Модуль}
Название класса. Module

Описание. Абстрактный класс для описания модуля


\subsection{Система сообщений}

{\bf Название класса.} RussianPost

\textit{Описание.} Используется два типа ящиков: входящие и исходящие. Первые служат для передачи сообщений конкретным модулям, а вторые обеспечивают массовую рассылку всем подписавшимся на ящик.

\textbf{Основные методы.} 

\begin{enumerate}

\item void addIncPostBox (String name, MsgRcvr mr)

Добавление исходящего ящика. В параметры передается имя ящика и получатель типа MsgRcvr
\item OutBox addOutPostBox(String name)

Добавление входящего ящика. В параметрах --- имя ящика.
\item void subscribe(String name, MsgRcvr self)

Подписка на существующий ящик. В аргументы передается имя ящика, и объект-подписчик типа MsgRcvr 
\item void send(String name, Object msg)

Передача сообщения на исходящий ящик. В аргументы передается имя ящика и сообщение.
\end{enumerate}

\subsection{Система настройки модулей}


Название класса. Configuration

Описание. Позволяет модулям хранить необходимые им данные и получать их по необходимости.
\begin{enumerate}
\item void add(String self, String objname, Object forserialize)

Добавление объекта для храниения. В аргументах: имя владельца, имя объекта, объект

\item Object get(String name, String objname)

Получить объект. Аргументы: имя владельца, имя объекта.
\item boolean register(String self)

Зарегистрироваться в системе настроект. Аргумент: имя модуля.
\end{enumerate}




\end{document}