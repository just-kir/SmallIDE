\documentclass[12pt]{article}


\usepackage[utf8]{inputenc}
\usepackage[russian]{babel}
\usepackage{amsmath}
\usepackage{amssymb}
\usepackage{geometry}
\geometry{top=2cm} %поле сверху
\geometry{bottom=2cm} %поле снизу
\geometry{left=2cm} %поле справа
\geometry{right=2cm}
\usepackage{graphicx}


\renewcommand{\rmdefault}{fac}

\begin{document}
\title{Текстовое задание для проекта, версия 0.1}
\author{Кирилл Павленко, Антон Шумаков}
\date{\today}
\maketitle
\newpage

\section{Цели}
Целью является создание интегрированной среды разработки (IDE) для комфортного программирования на языке Java. Задача имеет прикладной характер.

\section{Общая идея}
Мы ставим следующую задачу: создать IDE с максимально гибким и легко изменяемым интерфейсом для программистов Java. Чем наша среда разработки отличается от любой другой? В чем ее принцпиальное отличие? Идея заключается в том, что каждый функциональный элемент самой среды является модулем. Любой сторонний программист сможет легко переписать, модифицировать каждый модуль или добавить свой собственный модуль по своим предпочтениям. Это касается не только графических элементов, но и любых других, по нашей задумке изменениям может подвергнуться, повторюсь, любой функциональный элемент. Мы реализуем API для простого добавления модулей; будет доступен специальный режим IDE --- конструктор --- для разработки самой IDE. Наша схема позволит различным программистам обмениваться между собой модулями и сборками, что, в перспективе, поможет автономно существовать нашей среде разработки.

\section{Структура программы}
Мы реализуем три независимых компонента программы:
\begin{itemize}
\item Главное окно
\item Система обмена сообщениями
\item Система настройки
\end{itemize}
Эти три объекта являются единственными неизменяемыми элементами IDE. 

Главное окно представляет из себя поле для размещения графических элементов и выполняет функцию некоего $"$каркаса$"$ программы. 

Система сообщений необходима для взаимосвязи модулей между собой (о модулях --- позже). 

Система настройки --- структура для взаимодействия модулей с пользователем, позволяет настраивать ему их параметры.

Все остальные элементы программы --- отдельные модули. Некоторые из них --- графические --- взаимодействуют с пользователем чере главное окно. Другие, не являющиеся графическими элементами, могут $"$общаться$"$ друг с другом посредством системы сообщений. Пользователь может иметь к ним доступ через систему настройки. 

Для удобства использования мы придумали систему пакетов. Пакет --- это самодостаточный набор модулей, необходимый для корректной и полной работы IDE. По умолчанию в поставку будет входить минимальный стандартный пакет: текстовый редактор и кнопка компиляции. Возможно, мы расширим пакет к сдаче проекта. Пользователи смогут писать свои пакеты, заменяя или дополняя стандартный. Написание такого пакета будет осуществляться с помощью конструктора.

Мы организуем $"$Сборщик$"$, который позволит $"$компилировать$"$  набор пользовательских пакетов в уникальную IDE. 

Ниже представлена схема, соответствующая изложенному описанию.


\includegraphics[width=0.7\textwidth]{drawing.pdf}

\section{О реализации}

Проект будет полностью написан на языке Java, с использованием системы контроля версий --- git. Весь код будет доступен в свободном доступе на GitHub.com, там же будет отображаться прогресс нашей работы.

Ниже указаны очень приблизительные сроки работы. Заметим, что программа будет покрываться тестами по ходу написания. Более того, некоторые тесты будут написаны до того, как будет выполнен тестируемый модуль. Поэтому сроки включают в себе уже готовые, протестированные части.
\begin{itemize}
\item 5 марта. Начало работы.
\item 12 марта. Написание главного окна и его методов 
\item 22 марта. Реализация системы сообщений
\item 30 марта. Реализация методов системы настройки.
\item 10 апреля. Реализация сборщика модулей.
\item 15 апреля. Первая сборка IDE, минимальный пакет.
\item 25 апреля. Реализация конструктора для добавления модулей.
\item В остальное время будет произведена работа на графикой, по возможности будет расширен стандартный пакет.
\end{itemize}

Что касается распеределения по персоналиям, то дело обстоит следующим образом. Вся задача --- программистская. И мы оба будем заниматься исключительно программированием. И каждая задача будет реализована совместно, так как хорошо разибивается на достаточное количество классов. Поэтому под каждым пунктам подразумеваемся мы оба. Более того, на GIHub.com будет абсолютно прозрачно видно, кто какую часть работы выполнил (это входит в функционал данного сайта).

\section{Программное обеспечение}
IDE будет написана для операционной системы Linux, требуется предустановленная Java. 

\end{document}